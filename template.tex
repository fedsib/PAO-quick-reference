\chapter{Template}
I template sono usati per implementare il polimorfismo parametrico, in modo da rendere disponibile una programmazione generica sui tipi. Essi si dividono in:

\begin{itemize}
	\item Template di funzione
	\item Template di classe
\end{itemize}

\section{Template di funzione}
Un template di funzione rappresenta la descrizione di un metodo, per generare automaticamente istanze di funzione che differiscono per il tipo degli argomenti.

\lstinputlisting[language=C++,firstline=1,lastline=4]{src/template.cpp}

L'istanziazione implicita avviene invocando normalmente quella esplicita (che converte implicitamente i tipi dei parametri). I tipi della funzione, vengono dedotti dai tipi dei parametri attuali.

La sintassi per dichiarare un template è la seguente:

\textit{template <lista parametri>} ad es. \textit{template <class T, int size>}

I parametri di \textit{lista parametri} possono essere:

\begin{itemize}
	\item Parametri di tipo: preceduti dalle keyword \textit{class} o \textit{typename}
	\item Parametri valore: preceduti dal tipo del valore (int, nell'esempio precedente)
\end{itemize}

La deduzione dei tipi avviene tramite quattro tipi di conversioni:

\begin{itemize}
	\item Da L a R valore (T\& $ \rightarrow $ T)
	\item Da array a puntatore (T[] $ \rightarrow $ T\textasteriskcentered )
	\item Verso il const (T $ \rightarrow $ const T)
	\item Da valore a riferimento costante (T $ \rightarrow  $const T\&)
\end{itemize}

Il compilatore controlla i tipi da sinistra a destra.//

\textbf{NOTA:} Il compilatore memorizza le definizioni ma non le compila, genererà il codice macchina solo quando istanzierà i tipi giusti.

Vi sono due modi di compilazione

\begin{itemize}
	\item \textbf{Per inclusione:} Le definizioni dei template sono poste nei file header. Ciò viola il principio dell'\textit{Information Hiding} e genera codice duplicato, rallentando la compilazione; si piò forzare il tipo con una dichiarazione esplicita d'istanziazione  \textit{template int min(int,int)}. Usando la direttiva \textit{g++ -fno-implicit-templates} , verrà generato il codice solo per le istanziazioni esplicite.
	\item \textbf{Per separazione:} Le dichiarazioni dei template vanno poste in un file header, le definizioni in un altro file. La definizione va preceduta dalla keyword \textit{export}. Questa tipologia di compilazione non è supportata da tutti i compilatori. 
\end{itemize}

\section{Template di classe}

I template di classe sono utili per avere varie classi che differiscono per il tipo di campi dato, metodi e classi interne.

Occorre sempre un'istanziazione esplicita e i parametri possono avere valori di default.

La definizione del template è necessaria quando si deve operare su oggetti della classe (l'istanza serve per allocare lo spazio degli oggetti) , mentre non è necessaria per puntatori, riferimenti e dichiarazioni incomplete di classe.

\subsection{Metodi di template di classe}

E' possibile definire i metodi all'esterno della classe usando la sintassi:

\textit{Nome\_classe<Tipo>::Nome\_Metodo()\{\}}

\subsection{Dichiarazioni d'amicizia in template di classe}
In un template di classe possono apparire tre tipologie di dichiarazioni d'amicizia;

\begin{itemize}
	\item Dichiarazione di una classe (o funzione) friend non template (concreta) dentro ad un template di classe. Diventano amiche di istanze del template mentre le classi che contengono funzioni specifiche, usate nella classe template vanno dichiarate prima della classe templetizzata.
	\item Dichiarazione di un template di classe (o funzione) friend associato (ha tra i suoi parametri, alcuni dei parametri del template C). Tutte le istanze del template C hanno per amica una e una sola istanza del template associato. \`{E} il caso pi\`{u} comune, usa dichiarazioni incomplete. \textbf va prestata attenzione all'ordine di dichiarazione dei componenti.
	\item Amicizia con template non associati (gli insiemi dei parametri tr i due template sono disgiunti). Ogni dichiarazione dentro il template C va preceduta da: \textit{template <class Tp>}
\end{itemize}

\subsubsection{Esempi:}
Si rimanda agli esempi del libro a pag. 137-144

Un piccolo estratto, 

Per le dichiarazioni di template associati: Ad ogni istanza di \textit{QueueItem} è associata una sola istanza di amica della classe \textit{Queue}//

Per le dichiarazioni di template non associate: Tutte le istanze di \textit{Queue} amiche di ogni istanza di \textit{QueueItem} . Avremmo la classe \textit{Queue<int>} amica di \textit{QueueItem<string>}

\section{Membri statici in template}
Ogni istanza di T ha propri campi dati e metodi statici. 

\textbf{ATTENZIONE:} Ogni istanza significa, uno comune per ogni tipo.

L'inizializzazione del campo dati statico (istanziato solo se usato), è esterna alla classe, ad es. \textit{template <class T> int Queue<T>::contatore=0;}